
\documentclass[12pt]{article}
\setlength{\oddsidemargin}{0in}
\setlength{\evensidemargin}{0in}
\setlength{\textwidth}{6.5in}
\setlength{\parindent}{0in}
\setlength{\parskip}{\baselineskip}

\usepackage{amsmath,amsfonts,amssymb}

\title{CIS 545 - Final Project}
\author{Peter Kong}

\begin{document}

\maketitle



\hrulefill

\section{Introduction}
\subsection{Problem Domain}
\subsection{Problem Description}

\section{Data Acquisition \& Preparation}

\section{Exploration}

\section{Feature Engineering}

\section{Implementation}

\section{Analysis}
\subsection{Error Analysis}

We tried one variation of a sparse vector representation, which expanded the model to the top 1000 (instead of top 500) words.

% 500 window 3: Average Paired F-Score:  0.3670
% with .0001 weighting: 0.3657
% with .0001 weighting: 0.3646

%with 1000: .3681

% dense, SVD, 200 components (tried 100, same ballpark)
% .3569


We compare the impact of our model and clustering choices by evaluating their performance over the dev set (Table \ref{tab:sparseresults}):

\begin{table}[]
\centering
\begin{tabular}{l|l|l|l|l}
Vector Space Model	& KMeans	& Spectral		& Agglomerative	& Birch \\
\hline
500vec			& 0.3662	& 0.2876		& 0.3748			& 0.3649 \\
1000vec			& 0.3661	& 0.2739		& 0.3727			& 0.3652 \\        
\end{tabular}
\caption{Paired F-Score on the dev set by different vector space models and clustering algorithms.}
\label{tab:sparseresults}
\end{table}

\section{Dense Vector Representations}



\begin{table}[!h]
\centering
\begin{tabular}{l|l|l|l|l}
Dense Model	& KMeans	& Spectral		& Agglomerative	& Birch \\
\hline
SVD			& 0.3298	& 0.3165		& 0.3569			& 0.3076 \\
\end{tabular}
\caption{Paired F-Score on the dev set by different dense vector space models and clustering algorithms.}
\label{tab:denseresults}
\end{table}



\end{document}
